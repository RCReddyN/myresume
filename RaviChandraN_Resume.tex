\documentclass{article}

\usepackage{fontawesome5}
\usepackage{graphicx}
\usepackage{hyperref}
\usepackage{multicol}
\usepackage{geometry}

\geometry{
a4paper,
total={170mm,257mm},
left=20mm,
top=20mm,
}

\begin{document}

\section*{\Huge Ravi Chandra Reddy N}
\begin{tabular}{@{}lr}
\begin{tabular}{@{}lcr}
\faIcon{map-marker-alt} {Hyderabad, Telangana} & \faIcon{linkedin} \href{https://linkedin.com/in/rcreddyn}{rcreddyn} & \faIcon{globe} \href{https://rcreddyn.github.io}{rcreddyn.github.io}\\
\end{tabular}
\begin{tabular}{@{}lr}
\faIcon{phone} +919490977952 & \faIcon{envelope} \href{mailto:rcreddy1997@gmail.com}{rcreddy1997@gmail.com}\\
\end{tabular}
\end{tabular}

\section*{Education}
\faIcon{university} \textbf{BV Raju Institute of Technology (2014 - 2018)}
\begin{itemize}
\item \small Bachelor of Technology in Electrical and Electronics Engineering with 76\%.
\item \small Organizer for Technical, Cultural events and Industrial visits.
\item \small Supplementary education in embedded and assistive technology at Assistive Technology Lab.
\end{itemize}

\section*{Skills}
\small\textbf{Programming/Scripting Languages:} Java, Python, C/ C++, Bash \\
\small\textbf{Databases:} sqlite3, PostgreSQL, MongoDB \\
\small\textbf{Web development:} HTML/ CSS \\
\small\textbf{Misc:} Git, \LaTeX, Unix/ Linux, Software development, Algorithms, Data structures, Relational database, Non relational database, Object-oriented design, Mathematics, Operating systems, Computer networks

\section*{Coding profile(s)}
\begin{itemize}
\item \url {https://hackerrank.com/rcreddyn}
\end{itemize}

\section*{Experience}
\includegraphics[width=14px]{isro.png} \textbf{National Remote Sensing Center (2018)}
\begin{itemize}
\item \small Successfully delivered quality code during the internship as a Project Student in Bhuvan Geoportal and Web GIS Services.
\item \small Designed and developed standardized ways to communicate sensors and actuators with Bhuvan IoT Cloud.
\item \small Identified and sorted around 100 sensors and actuators into categories, to generalize a design solution.
\item \small Delivered Python scripts and equivalent C++ code to efficiently interface devices with development boards.
\item \small Formatted timed sensory data as JSON Objects to send to the IoT Cloud.
\end{itemize}

\section*{Projects}
\href{https://github.com/rcreddyn/nomsh}{\textbf{Nomsh}}, a shell programmed in C.
\begin{itemize}
\item Effectively implemented support for execution of built-in, executable commands, and output redirection.
\item Written a Makefile for creating an executable on the Linux platform.
\end{itemize}
\href{https://github.com/rcreddyn/nomsh}{\textbf{Amrika}}, a compiler engineered for a cooked-up language.
\begin{itemize}
\item Crafted grammar for a language and implemented parser, lexer, and emitter from scratch.
\item Written a makefile to generate the compiler, and a bash script to run the compiled code.
\end{itemize}
\href{https://github.com/rcreddyn/nomsh}{\textbf{Lavangam}}, a trie-based command-line spell checker tool.
\begin{itemize}
\item Engineered a utility that scans for a keyword and suggests a list if the keyword is misspelt.
\item Successfully implemented a trie to store/search for words.
\item Utilized Levenshtein's algorithm to calculate the distance between two words, which significantly improved performance.
\item Utilized wordlist from NLTK corpus.
\end{itemize}
\href{https://github.com/rcreddyn/nomsh}{\textbf{Holyperil}}, a command-line planetary monitoring application.
\begin{itemize}
\item A Python script utilizing NASA's Asteroids - NeoWs API, to notify of potentially hazardous/ dangerous asteroids approaching Earth.
\item Used an offset of six days, to prepare just in case of approaching hypothetical danger.
\end{itemize}
\end{document}