\documentclass{article}

\usepackage{fontawesome5}
\usepackage{graphicx}
\usepackage{hyperref}
\begin{document}

\section*{\Huge Ravi Chandra Reddy N}

\begin{tabular}{ll}
\faIcon{phone} +919490977952 &\faIcon{envelope} rcreddy1997@gmail.com\\
\faIcon{linkedin} \url{https://linkedin.com/in/rcreddyn}& \faIcon{github} \url{https://github.com/rcreddyn}\\
\end{tabular}


\section*{Education}
\faIcon{university} BV Raju Institute of Technology (2014 - 2018)
\begin{itemize}
\item \small Bachelor of Technology in Electrical and Electronics Engineering.
\item \small Organizer for Technical, Cultural events and Industrial visits.
\item \small Supplementary education in Embedded and Assistive Technology  at Assitive Technology Lab.
\end{itemize}

\section*{Skills}
\begin{itemize}
\begin{minipage}{0.5\linewidth}
    \item Java, C, and Basic C++
    \item  Python, and Shell Scripting
    \item Algorithms, and Data Structures
\end{minipage}
\begin{minipage}{0.4\linewidth}
    \item PostgreSQL, Git
    \item HTML, CSS
    \item \LaTeX
\end{minipage}
\end{itemize}

\section*{Coding profile(s)}
\begin{itemize}
\item \href {https://hackerrank.com/rcreddyn}{HackerRank}
\item \href {https://auth.geeksforgeeks.org/user/rcreddyn/}{GeeksforGeeks}
\end{itemize}

\section*{Experience}
\includegraphics[width=14px]{isro.png} National Remote Sensing Center (2018)
\begin{itemize}
\item \small Project Student
\item \small Designed and developed standard ways to trans-receive data from sensors and actuators to  Bhuvan IoT Cloud.
\end{itemize}

\section*{Projects}
\begin{itemize}
\item \href{https://github.com/rcreddyn/nomsh}{Nomsh}, a shell written in C.
\begin{itemize}
\item Supports execution of builtin, and executable commands.
\item Supports output redirection.
\item Wrote a Makefile for creating an executable.
\end{itemize}
\item  \href{https://github.com/rcreddyn/nomsh}{Amrika}, a compiler written in C++ for a cooked up language.
\begin{itemize}
\item Wrote grammar for a language, that draws from popular languages and implemented parser, lexer, and emiter for the same from scratch.
\item Implemented print and assignment statements and analogues to if conditional and while iteration.
\item Wrote a makefile to generate the compiler, and a bash script to run the compiled code.
\end{itemize}
\item \href{https://github.com/rcreddyn/nomsh}{Lavangam}, a trie-based command line spell checker written in Python.
\begin{itemize}
\item Implemented a trie to store/search for words.
\item Implemented Levenshtein's algorithm to calculate distance between two words.
\item Utilized wordlist from NLTK corpus.
\end{itemize}
\item \href{https://github.com/rcreddyn/nomsh}{Holyperil}, a command line planetory monitering application written in Python. 
\begin{itemize}
\item Code written in Python to notify potentially hazardous/ dangerous asteroids approaching Earth.
\item Used an offset of six days, to prepare just in case of an approaching hypothetical danger.
\item Utilized NASA's Asteroids - NeoWs API for the search.
\end{itemize}
\item \href{https://github.com/rcreddyn/nomsh}{Freddie}, a text-only browser.
\begin{itemize}
\item Code written in Python to extract text from webpages, bypassing HTML tags, CSS, and all kinds of media.
\end{itemize}
\end{itemize} 
\end{document}